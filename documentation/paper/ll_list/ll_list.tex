\documentclass{article}
\usepackage{amsmath}
\usepackage{amssymb}
\usepackage{stmaryrd}
\usepackage{gcl}
\usepackage[inlcmnts]{flagderiv}
\usepackage{pascal}
\usepackage{varioref}
\usepackage{ifthen}


% Some logic operators.
\newcommand{\eqval}{\ensuremath{\mathrel{\stackrel{val}{=}}}}
\newcommand{\neqval}{\ensuremath{\mathrel{\stackrel{val}{\neq}}}}
\newcommand{\modelsval}{\ensuremath{\mathrel{\stackrel{val}{\models}}}}
\renewcommand{\implies}{\ensuremath{\mathrel{\Rightarrow}}}
\newcommand{\rimplies}{\ensuremath{\mathrel{\Leftarrow}}}
\newcommand{\biimplies}{\ensuremath{\mathrel{\Leftrightarrow}}}

% Hoare triple: { state-predicate }  transition  { state-predicate }
\newcommand{\hoaretrip}[3]{
	\ensuremath{
		\begin{array}{ccc}
			\state{#1} & #2 & \state{#3}
		\end{array}
	}
    }
\newcommand{\hoaretripNL}[3]{
	\ensuremath{
		\begin{aligned}
			&\state{#1} \\ &#2 \\ &\state{#3}
		\end{aligned}
	}
    }

% Declare a flag type that models the assignment stage in a Hoare triple
\newcommand{\transformsymb}{\textbf{Transformation}:}
\newcommand{\transformation}[3]{\assume{#1}{\textbf{\transformsymb}~#2}{#3}}


\newcommand{\nil}{\textbf{nil}}
\newcommand{\incref}[1]{\texttt{incref}(\deref{#1})}
\newcommand{\decref}[1]{\texttt{decref}(\deref{#1})}
\newcommand{\atomic}[1]{\textbf{atomic}\langle #1 \rangle}
\newcommand{\AtomVar}[1]{\overline{#1}}
\newcommand{\AtomRead}[2]{\atomic{#1 \become \AtomVar{#2}}}
\newcommand{\AtomExchange}[3]{\atomic{#1, \AtomVar{#2} \become \AtomVar{#2}, #3}}
\newcommand{\CAS}[3]{\atomic{\textrm{cas}: \AtomVar{#1}, #2, #3}}
\newcommand{\reachable}[2]{\left[#2 \nnearrow #1\right]}
\newcommand{\true}{\textrm{true}}
\newcommand{\false}{\textrm{false}}
\newcommand{\True}{\textrm{True}}
\newcommand{\False}{\textrm{False}}
\newcommand{\atomptr}{ll\_pointer}
\newcommand{\deref}[1]{\overrightarrow{#1}}
\newcommand{\allthreads}{T}
\newcommand{\thread}{\tau}
\newcommand{\otherthread}{\tau^+}
\newcommand{\curthread}{\tau_0}
\newcommand{\qthread}{\tau_q}
\newcommand{\yields}{\rightarrow}	% state = transition \yields new_state
\newcommand{\choice}{\sqcap}		% state = ts_1 or ts_2
\newcommand{\state}[1]{\left\llbracket #1 \right\rrbracket}  % state with given predicate


\begin{document}
\title{Lock-free linked list implementation}
\author{Ariane van der Steldt\\
    \texttt{ariane@stack.nl}}
\date{\today}
\maketitle



\begin{abstract}
An algorithm for a doubly-linked list,
allowing concurrent, wait-free insert/update/delete operations.
\end{abstract}



\section{Definitions}



\subsection{Threading}
\begin{subequations}
$\allthreads$ denotes the set of threads in the universe.
Elements of $\allthreads$ are denoted with $\thread$:
\begin{equation}
\thread \in \allthreads
\end{equation}
$\curthread$ denotes the current thread:
\begin{equation}
\curthread \in \allthreads
\end{equation}
$\otherthread$ denotes any thread excluding the current thread:
\begin{equation}
\otherthread \in \allthreads \land \otherthread \neq \curthread
\end{equation}
\end{subequations}



\subsection{Transition notation}
\begin{subequations}
The notation
\begin{equation}
\hoaretrip{P_0}{\lambda}{P_1}
\end{equation}
denotes the transition
from state $P_0$ to state $P_1$, using the transformation $\lambda$.

The notation
\begin{equation}
P_0 = (\lambda_1 \yields P_1) \choice (\lambda_2 \yields P_2)
\end{equation}
denotes a state $P_0$, where either transition $\lambda_1$ yields state $P_1$,
or transition $\lambda_2$ yields state $P_2$.

The states may be noted as predicates inside double square brackets, thus:
\begin{equation}
\state{p = q} \models p = q
\end{equation}
denotes that state $\state{p = q}$ models a state
where the predicate $p = q$ evaluates $\true$.
Note that the state-predicate may be stronger than the predicate it models:
\begin{equation}
\state{p = 1 \land q = 1} \models p = q
\end{equation}
Note that the notation is distinct from the predicate it contains:
\begin{equation}
\left( \state{p = q} \equiv \left(p = q\right) \right) \implies \false
\end{equation}
since a state contains all predicates where the state-predicate holds.
Formally:
\begin{equation}
S = \state{\mathcal{P}} \equiv
    S \in \left\{ s \  | \ \textrm{predicate of $s$} \modelsval \mathcal{P}
    \right\}
\end{equation}

Note also that states hold predicate logic:
\begin{equation}
\left( \state{\mathcal{P}} \models \state{\mathcal{Q}} \right) \biimplies
\left( \mathcal{P} \modelsval \mathcal{Q} \right)
\end{equation}
\end{subequations}



\subsection{Abusing flag derivations for transition proof}
In this paper, I will be (ab)using flag derivations to proof transitions.
\begin{flagderiv}
\introduce{abuse:intro}
    {\mathcal{S}: \mbox{state}}
    {Introduce a predicate $\mathcal{S}$}
	\skipsteps{2}{\vdots}{}
	\step{}
	    {\mathcal{S} \implies \mathcal{A}}
	    {Some predicate $\mathcal{A}$}
	\transformation{abuse:transition}
	    {\lambda \yields \mathcal{T}}
	    {Introduce transformation $\lambda$ on $\mathcal{S}$}
		\skipsteps{2}{\vdots}{}
		\step{}
		    {\mathcal{T} \implies \mathcal{B}}
		    {Derive some predicate $\mathcal{B}$ from $\mathcal{T}$}
		\step{}
		    {\mathcal{A} \land \mathcal{T} \rimplies \mathcal{A}}
		    {Only way to pull in $\mathcal{A}$
		      is to filter through $\mathcal{T}$}
		\step{abuse:conclusion}
		    {
			\mathcal{B} \land
			\left(
				\mathcal{A} \land \mathcal{T}
				\rimplies \mathcal{A}
			\right)
		    }
		    {Final conclusion on
		      $\mathcal{S} \yields \lambda \yields \mathcal{T}$}
\conclude[2]{}
    {
	\hoaretrip
	    {\mathcal{S}}
	    {\lambda}
	    {
		\mathcal{B} \land
		\left(
			\mathcal{A} \land \mathcal{T}
			\rimplies \mathcal{A}
		\right)
	    }
    }
    {Create Hoare triple using
      \ref{abuse:intro}, \ref{abuse:transition}, \ref{abuse:conclusion}}
\end{flagderiv}
Every predicate $\mathcal{P}$ during the transformation flag
\ref{abuse:transition} is only valid if:
$
\left(
	\mathcal{T} \implies \mathcal{P}
\right)
\lor
\left(
	\left( \mathcal{S} \implies \mathcal{P} \right)
	\land
	\left( \mathcal{P} \land \mathcal{T} \equiv \true \right)
\right)
$.
In other words, predicate $\mathcal{T}$ acts as a filter on $\mathcal{S}$.
Because of this, extra care is required to ensure that $\mathcal{T}$
is complete in describing all modifications introduced by $\lambda$.



\subsection{Dereferencing}
\begin{subequations}
A pointer $p$ may refer to a value $V$, which is written as:
\begin{equation}
\deref{p} = V
\end{equation}
Dereferencing need not be an atomic operation.

Dereferencing $\nil$ is malformed:
\begin{equation}
\left( \exists_v \deref{p} = v \right) \biimplies p \neq \nil
\end{equation}

The dereference operation is a bijection:
\begin{equation}
a \equiv b \biimplies \deref{a} \equiv \deref{b}
\end{equation}
\end{subequations}



\subsection{Reachability}
\begin{subequations}
Reachability is defined as the property on $V$
such that at least one variable $v$ exists, where $\deref{v} = V$
for which $v$ has shared ownership on $V$.
This relation is written as $\reachable{V}{v}$.
\begin{equation}
\label{reachability-deref}
\reachable{V}{v} \implies \deref{v} = V
\end{equation}
%
As a special case, thread-reachability is defined as
a thread having at least a single variable ($\aleph$) that maintains
reachability.
\begin{equation}
\reachable{V}{\thread} \implies
    \left(\exists_{\aleph \in \thread} \deref{\aleph} = V\right)
\end{equation}
%
Reachability is transferable:
\begin{equation}
p = q \implies \left(
\reachable{\deref{p}}{p} = c_0 + 1 \land \reachable{\deref{q}}{q} = c_1
\biimplies
\reachable{\deref{p}}{p} = c_0 \land \reachable{\deref{q}}{q} = c_1 + 1
\right)
\end{equation}
%
Naturally, reachability may not fall below $0$:
\begin{equation}
\reachable{V}{v} \ge 0
\end{equation}
The notation $\neg \reachable{V}{v}$ denotes that
$v$ holds no references to $V$:
\begin{equation}
\label{reachability-neg}
\neg \reachable{V}{v} \equiv \reachable{V}{v} = 0
\end{equation}
Note that in this case, $\deref{v} \neq V \lor \deref{v} = V$.
Similarly, the notation
$\reachable{V}{v}$ (reachability as a predicate)
is short-hand for $\reachable{V}{v} = 1$:
\begin{equation}
\label{reachability-one}
\reachable{V}{v} \equiv \reachable{V}{v} = 1
\end{equation}

As a special notation, a group of variables can be sharing a given
reachability:
\begin{equation}
\label{reachability-shared}
\reachable{V}{a, b, c...} = N \equiv
\reachable{V}{a} + \reachable{V}{b, c...} = N
\end{equation}
%
Reachability on $\nil$ is malformed:
\begin{equation}
\label{reachability-nil}
\reachable{P}{p} \neq 0 \implies p \neq \nil
\end{equation}

Reachability constraints can be
weakened\footnote{See proof \vref{proof-reachability-weakening}.}:
\begin{equation}
\label{reachability-weakening}
\neg\reachable{\mathcal{O}}{a} \land \reachable{\mathcal{O}}{b} \implies
\reachable{\mathcal{O}}{a, b}
\end{equation}
\end{subequations}



\section{Algorithm}



\section{Proof derivations}


\subsection{Proof: reachability weakening}
\label{proof-reachability-weakening}
Proof for formula \vref{reachability-weakening}.
\begin{flagderiv}
\introduce{intro-abO}
    {a, b, \mathcal{O}:
	\neg\reachable{\mathcal{O}}{a} \land
	\reachable{\mathcal{O}}{b}}
    {Introduce precondition.}
	\step{abO-as-arithmatic}
	    {\reachable{\mathcal{O}}{a} = 0 \land
	      \reachable{\mathcal{O}}{b} = 1}
	    {Definition of reachability on \ref{intro-abO}:
	      formula \ref{reachability-neg} and \ref{reachability-one}}
	\step{abO-eq-one}
	    {\reachable{\mathcal{O}}{a} + \reachable{\mathcal{O}}{b} = 1}
	    {Arithmatic on \ref{abO-as-arithmatic}}
	\step{abO-shared}
	    {\reachable{\mathcal{O}}{a, b} = 1}
	    {Shared reachability equivalence on \ref{abO-eq-one}:
	      formula \ref{reachability-shared}}
\conclude{}
    {
	\forall_{a, b, \mathcal{O}}
	    \left(
		\neg\reachable{\mathcal{O}}{a} \land \reachable{\mathcal{O}}{b}
		\implies
		\reachable{\mathcal{O}}{a, b}
	    \right)
    }
    {$\forall$-intro on \ref{intro-abO} and \ref{abO-shared}}
\end{flagderiv}



\end{document}
